\documentclass[letterpaper,11pt]{article}

\usepackage[T1]{fontenc}
\usepackage[sc,osf]{mathpazo}
\usepackage[hmargin=1.5cm,vmargin=1.5cm]{geometry}
\usepackage{hyperref}

% Don't indent paragraphs
\setlength\parindent{0em}
\setlength{\parskip}{1em}

% Custom section fonts
\usepackage{sectsty}
\sectionfont{\rmfamily\mdseries\Large}
\subsectionfont{\rmfamily\mdseries\itshape\large}

\begin{document}

\vspace*{2\baselineskip}

Dear Selection Committee,

I am writing to apply for the 2025 Panofsky Fellowship at SLAC National Accelerator Laboratory. My research program combines physics-informed machine learning approaches for real-time analysis at light sources with innovative frameworks for natural language agent-assisted automation of beamline workflows. This dual focus addresses critical challenges in modern light source science while advancing SLAC's mission of enabling cutting-edge research through technical innovation.

My work at SLAC has already demonstrated the potential of physics-informed machine learning through PtychoPINN, a framework that runs three orders of magnitude times faster than traditional phase retrieval methods while maintaining accuracy. This innovation has catalyzed collaborations with groups at the Advanced Photon Source and Lawrence Berkeley Laboratory, who are now incorporating these methods into their analysis frameworks. This adoption across multiple national laboratories highlights both the broad impact of this approach and my ability to lead collaborative development efforts that benefit the wider scientific community.

The proposed research program builds on these successes while pushing into new territory. The physics-informed ML component extends proven methods to address pressing needs in real-time analysis across SLAC's experimental facilities. The AI-assisted automation framework represents a more exploratory direction that could transform how users interact with increasingly complex analysis and data collection workflows. This combination of near-term technical advances with longer-term innovation aligns with SLAC's dual commitments to immediate scientific productivity and long-range facility development.

My background positions me to execute this research program. Through developing practical scientific machine learning methods and collaborating with LCLS scientists, I've gained direct experience with the technical and operational challenges of light source experiments. My work on probabilistic modeling, from physics-informed neural networks to variational inference and hierarchical Bayesian modeling, demonstrates the mathematical preparation needed to advance these methods. Equally important is my proven ability to bridge theoretical innovation with practical implementation, as shown by the successful deployment and adoption of my tools across multiple facilities.

The Panofsky Fellowship would provide the ideal platform to pursue this research program, combining independent investigation with the collaborative opportunities that make SLAC an exceptional environment for scientific innovation. I look forward to discussing how my research program could contribute to SLAC's continuing leadership in light source science.

Thank you for your consideration.

\vspace{\baselineskip}

Sincerely,\\
Oliver Hoidn

\end{document}
